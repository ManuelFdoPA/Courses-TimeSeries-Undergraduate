\documentclass[12] {article}
\usepackage{setspace}
\usepackage{amssymb}


\begin{document}
\onehalfspace

\title{Lecture 2-3}
\date{}
\maketitle

\section{Main Definitions}

\textbf{Definition 1:} The \underline{mean function} of a real-valued time series is the function
\[ \mu_{X}: T \rightarrow \mathbb{R} \]
defined by:
\[ \mu_{X}(t) \equiv E[X_t]\]

The mean function is one of the features of a time series model that allows us to understand how the marginal distributions of a time series changes over time.  \\


\noindent \textbf{Definition 2:} Two random variables $X,Y$ with a joint p.d.f. $f(x,y)$ are \underline{independent }if the joint p.d.f. $f$ can be written as:
\[ f(x,y) = f_1(x) f_2(y)  \]  
where $f_1$ and $f_2$ are two non-negative functions that integrate to one. Two random variables are said to be \emph{dependent} if they are not independent. \\

\noindent Here are some properties of independent random variables: \\

\noindent \textbf{Claim:} For any sets $A, B$ 
\[ \mathbb{P} (X \in A \textrm{ and } Y \in B) = \mathbb{P} (X \in A) \mathbb{P} (Y \in B)   \]

\noindent Proof:

\begin{eqnarray*}
\mathbb{P} (X \in A \textrm{ and } Y \in B) &=& \int_{A} \left( \int_{B}  f(x,y) dy \right) dx \\
&= & \int_{A} \left( \int_{B}  f_1(x) f_2(y) dy \right) dx \\
&=&  \int_{A} \left( \int_{B}   f_2(y) dy \right)  f_1 (x) dx \\
&=& \mathbb{P} (X \in A) \mathbb{P} (Y \in B).
\end{eqnarray*}

This means that the product of any joint event, can be written as the product of the marginal events.  \\

\textbf{Corollary:} $\mathbb{P}(X \in A | Y \in B) = \mathbb{P}(X \in A)$ \\

How do we know if two random variables are dependent? Well, based on the definition above this means that all we need to do is find a set $A$ and a set $B$ for which:

\[ \mathbb{P}(X \in A | Y \in B) \neq \mathbb{P}(X \in A) \]

Computing conditional probabilities is difficult. But here is a trick we can use. Remember that for any two random variables $X$ and $Y$ the covariance is defined as:

\[ \textrm{Cov}(X,Y) \equiv E [ (X-\mu_x)(Y-\mu_y) ]  \]


\noindent \textbf{Claim:} If $(X, Y)$ are independent then $\textrm{Cov}(X,Y)=0$.  \\


\noindent Proof: For any functions $g,h$:\\
\begin{eqnarray*}
E[ g(x) h(y) ] &=& \int_{\mathbb{R}} \left( \int_{\mathbb{R}} g(x) h(y) f(x,y) dx \right) dy \\
&=& \int_{\mathbb{R}} \left( \int_{\mathbb{R}} g(x) h(y) f_1(x) f_{2}(y) dx \right) dy \\
&=& \int_{\mathbb{R}} \left( \int_{\mathbb{R}} g(x)  f_1(x)  dx \right) h(y) f_{2}(y) dy \\
&=& E[ g(x) ] E[h(y)]
\end{eqnarray*}
Taking $g(x)=x -\mu_x$ completes the proof, as $E[x-\mu_x]=0$.  \\


This means that in order to show that two random variables are not independent, it suffices to find a non-zero covariance. Thus, the covariance can give us some superficial and preliminary glimpse into the dependence embedded in a time series model.  \\

\noindent \textbf{Definition 3:} The \underline{covariance function} of a real-valued time series is the function:
\[ \gamma_{X}: T \times T \rightarrow \mathbb{R} \]
defined by:
\[ \gamma_{X}(s,t) \equiv \textrm{Cov} ( X_s, X_t )\] 

\noindent The \underline{autocorrelation function} is simply defined as

\[  \rho_{X}(r,s) = \gamma_X(r,s) / \sqrt{\gamma_{X}(r,r)} \sqrt{\gamma_{X}(s,s)}   \]

Let us compute the covariance function for the cosine model. Note that:

\begin{eqnarray*}
\gamma_{X} (r,s) &=& \mathbb{E}  [ (X_r - \mu_X(r)) (X_s - \mu_X(s))    ]   \\
&=& E [ \varepsilon_{r} \varepsilon_s ] \\
&=& 0
\end{eqnarray*}

In general, zero covariance does not imply independence (remember this?). However, if the joint distribution is normal, then it does. \\

Complement these notes with Section 1.4 in the book. 

\end{document}

\newpage

Let us introduce a less boring model. Suppose that you have 
\[ P ( \varepsilon_{t} = 1) = 1/2 \quad P ( \varepsilon_{t} = -1) = 1/2 \]

Consider the model:

\[ X_t = \sum_{j=1}^{t} \varepsilon_t \]

Let us do the same thing that we did before. Let us first analyze the marginals and then simulate it. \\

What is the mean function? What is the covariance function?\\


\noindent \textbf{Definition 3:} A times series is \underline{weakly stationary} if \\

i) The mean function does not depend on $t$ \\

ii) $\gamma_{X}(t+h,t)$ is independent of $t$ for each $h$. \\

This means that if $X$ is a stationary model we can write the autocovariance function as a function of $h$ only:

\[ \gamma_{X}(h) \equiv \gamma_{X} (t+ h, t) = E[(X_{t+h} - \mu)(X_t - \mu) ]\]

How do we estimate the mean function and the autocovariance function of a time series.


	
Let us compute the covariance function for the cosine model. Note that:

\begin{eqnarray*}
\gamma_{X} (r,s) &=& \mathbb{E}  [ (X_r - \mu_X(r)) (X_s - \mu_X(s))    ]   \\
&=& E [ \varepsilon_{r} \varepsilon_s ] \\
&=& 0
\end{eqnarray*}

In general, zero covariance does not imply independence (remember this?). However, if the joint distribution is normal, then it does. \\

Complement these notes with Section 1.4 in the book. 

\end{document}

\newpage

Let us introduce a less boring model. Suppose that you have 
\[ P ( \varepsilon_{t} = 1) = 1/2 \quad P ( \varepsilon_{t} = -1) = 1/2 \]

Consider the model:

\[ X_t = \sum_{j=1}^{t} \varepsilon_t \]

Let us do the same thing that we did before. Let us first analyze the marginals and then simulate it. \\

What is the mean function? What is the covariance function?\\


\noindent \textbf{Definition 3:} A times series is \underline{weakly stationary} if \\

i) The mean function does not depend on $t$ \\

ii) $\gamma_{X}(t+h,t)$ is independent of $t$ for each $h$. \\

This means that if $X$ is a stationary model we can write the autocovariance function as a function of $h$ only:

\[ \gamma_{X}(h) \equiv \gamma_{X} (t+ h, t) = E[(X_{t+h} - \mu)(X_t - \mu) ]\]

How do we estimate the mean function and the autocovariance function of a time series.





\end{document}